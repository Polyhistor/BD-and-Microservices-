\documentclass[conference]{IEEEtran}
\IEEEoverridecommandlockouts
% The preceding line is only needed to identify funding in the first footnote. If that is unneeded, please comment it out.
\usepackage{cite}
\usepackage{amsmath,amssymb,amsfonts}
\usepackage{algorithmic}
\usepackage{graphicx}
\usepackage{textcomp}
\usepackage{xcolor}
\def\BibTeX{{\rm B\kern-.05em{\sc i\kern-.025em b}\kern-.08em
    T\kern-.1667em\lower.7ex\hbox{E}\kern-.125emX}}
\begin{document}

\title{Conference Paper Title*\\
{\footnotesize \textsuperscript{*}Note: Sub-titles are not captured in Xplore and
should not be used}
\thanks{Identify applicable funding agency here. If none, delete this.}
}

\author{\IEEEauthorblockN{1\textsuperscript{st} Given Name Surname}
\IEEEauthorblockA{\textit{dept. name of organization (of Aff.)} \\
\textit{name of organization (of Aff.)}\\
City, Country \\
email address or ORCID}
\and
\IEEEauthorblockN{2\textsuperscript{nd} Given Name Surname}
\IEEEauthorblockA{\textit{dept. name of organization (of Aff.)} \\
\textit{name of organization (of Aff.)}\\
City, Country \\
email address or ORCID}
\and
\IEEEauthorblockN{3\textsuperscript{rd} Given Name Surname}
\IEEEauthorblockA{\textit{dept. name of organization (of Aff.)} \\
\textit{name of organization (of Aff.)}\\
City, Country \\
email address or ORCID}
\and
\IEEEauthorblockN{4\textsuperscript{th} Given Name Surname}
\IEEEauthorblockA{\textit{dept. name of organization (of Aff.)} \\
\textit{name of organization (of Aff.)}\\
City, Country \\
email address or ORCID}
\and
\IEEEauthorblockN{5\textsuperscript{th} Given Name Surname}
\IEEEauthorblockA{\textit{dept. name of organization (of Aff.)} \\
\textit{name of organization (of Aff.)}\\
City, Country \\
email address or ORCID}
\and
\IEEEauthorblockN{6\textsuperscript{th} Given Name Surname}
\IEEEauthorblockA{\textit{dept. name of organization (of Aff.)} \\
\textit{name of organization (of Aff.)}\\
City, Country \\
email address or ORCID}
}

\maketitle

\begin{abstract}
This document is a model and instructions for \LaTeX.
This and the IEEEtran.cls file define the components of your paper [title, text, heads, etc.]. *CRITICAL: Do Not Use Symbols, Special Characters, Footnotes, 
or Math in Paper Title or Abstract.
\end{abstract}

\begin{IEEEkeywords}
component, formatting, style, styling, insert
\end{IEEEkeywords}

\section{Requirements Specification}

Precursor to theorizing about the potential of microservices patterns for big data systems, we need to define what we mean by big data systems and what are the requirements of these systems. System and software requirements come in different flavour and can range from a sketch on a napkin to formal (mathematical) specifications. Therefore, we first need to identify what kind of requirements is the most suitable for the purposes of this study. To answer this question, we first explored the body of evidence to understand the current classification of software requirements. 

There's been various attempts to defining and classifying software and systems requirements. For instance, Sommerville (\cite{sommerville2011software}) classified requirements into three levels of abstraction that are namely 1) user requirements, 2) system requirements and 3) design specifications. The author then mapped these requirements against user acceptance testing, integration testing and unit testing. While this could satisfy the requirements of this study, we opted for a a more general framework provided by Laplante (\cite{laplante2017requirements}). In Laplante's approach, requirements are categorized into three categories of 1) functional requirements, 2) non-functional requirements, and 3) domain requirements. 

Our objective is to define the high-level requirements of big data systems, thus we do not seek to explore 'non-functional' requirements. Non-functional requirements are emerged from the particularities of an environment, such as a banking sector and do not correlate to our study. 

After clarifying the type of requirements, we then explored the body of evidence to realize the general requirements of big data systems. Indeed, the most discussed characteristics of big data systems are the popular 5Vs which are velocity, veracity, volume, Variety and Value (\cite{Demchenko2014}, \cite{Bughin2016}, \cite{Bahrami2015}, \cite{rad2017big}, \cite{Marz2015}, \cite{Chen2016a} ). Many researchers such as Nadal et al (\cite{nadal2017software}) have underpinned their artifact development on these characteristics and requirements that emerge from them. 

Along the lines, we 




\bibliographystyle{IEEEtran}
\bibliography{references}

\end{document}
